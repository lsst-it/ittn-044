\documentclass[PMO,authoryear,toc]{lsstdoc}
% lsstdoc documentation: https://lsst-texmf.lsst.io/lsstdoc.html
%\input{meta}

% Package imports go here.

% Local commands go here.

%If you want glossaries
%\input{aglossary.tex}
%\makeglossaries

\title{LHN Specifications and Design Documents Catalog}

% Optional subtitle
% \setDocSubtitle{A subtitle}

\author{%
Cristian Silva
}

\date{\today}

\setDocRef{ITTN-044}
\setDocUpstreamLocation{\url{https://github.com/lsst-it/ittn-044}}
%\setDocDate{2021-04-14}
%\date{\vcsDate}

% Optional: name of the document's curator
% \setDocCurator{The Curator of this Document}

\setDocAbstract{%
Hub of documentation related with the design and specifications of the long haul network
}

% Change history defined here.
% Order: oldest first.
% Fields: VERSION, DATE, DESCRIPTION, OWNER NAME.
% See LPM-51 for version number policy.
\setDocChangeRecord{%
  \addtohist{1}{2021-04-14}{First version, dump from Confluence.}{Cristian Silva}
}


\begin{document}

% Create the title page.
\maketitle
% Frequently for a technote we do not want a title page  uncomment this to remove the title page and changelog.
% use \mkshorttitle to remove the extra pages


\section{Documents Links}

Data Management System-level network specifications (extracted from \href{https://confluence.lsstcorp.org/download/attachments/20284335/LSE-61.pdf?version=1&modificationDate=1490879770000&api=v2}{LSE-61 Data Management Subsystem Requirements})

180 days MTBF/year

48 hours MTTR/year

Base to Archive nightly data volume (science images and meta-data): 15TB

Amount of time available to transfer data from Mountain to Base to Archive:

                 Crosstalk-corrected images for Alert Production: 6 seconds to move (between 2.6 - 12.4 Gbytes depending on compression)

                 Raw images: 24 hours

Note, there is also Observatory Control System and other operational data transferred at night, and there is daytime engineering and calibration traffic potentially at the same volume.  Finally, there is periodic data transfer from the Archive to the Base, primarily as a result of annual Data Release Processing.  All the data flows and allocated bandwidths are defined in the following documents.

A grouping of this traffic for purposes of network design, QoS, prioritization etc. is on the \href{https://confluence.lsstcorp.org/display/DM/LSST+Network+Traffic+Types}{LSST Network Traffic Types}.

A bandwidth allocation by link of this traffic for purposes of network design, QoS, prioritization etc. is on the \href{https://confluence.lsstcorp.org/display/DM/LSST+Network+Bandwidth+Allocation}{LSST Bandwidth Allocation}.

The LSST Long-Haul Networks are specified in \href{https://docushare.lsstcorp.org/docushare/dsweb/Get/LSE-78/lse78observatoryNetworkDesign_rel5.1_20200825.pdf}{LSE-78 Rubin Observatory Network Design} and \href{https://docushare.lsstcorp.org/docushare/dsweb/Get/LSE-479/lse479observatoryNetworkTechnicalDoc_rel1_20200825.pdf}{LSE-479 Rubin Observatory Networks Technical Document}.

The budgeted cost and schedule of deployment of links and bandwidth, and their utilization is documented in \href{https://confluence.lsstcorp.org/download/attachments/20284335/20170130%20LDM-142%20LSST%20Networks%20BL%20and%20Plan.xls?version=1&modificationDate=1491479508000&api=v2}{LDM-142 Network Sizing Model}

The requirements for the Summit Network are in \href{https://docushare.lsstcorp.org/docushare/dsweb/Get/LTS-577/LTS-577%20Summit%20Network%20Specification%20Rel1%2006292017.pdf}{LTS-577 Summit Network Specification}

In addition to the top-level documents above, there are documents describing how the LSST Networks are to be tested, verified, and managed:

The \href{https://confluence.lsstcorp.org/download/attachments/20284335/LSST%20LHN%20End-to-End_Plan_v6.docx?version=1&modificationDate=1490879785000&api=v2}{LSST Network End-to-End Test Plan} defines the plan for development testing and monitoring the LSST networks.

The \href{https://docushare.lsstcorp.org/docushare/dsweb/Get/LDM-732/NoContent8135677362395171770.txt}{Vera C. Rubin Network Verification Baseline} defines the plan for formal verification of the LSST Networks

\href{https://docushare.lsstcorp.org/docushare/dsweb/Get/Document-35934/Rubin%20Observatory%20Networks%20Pre-Verification%20Review%20Report%202020-06-22.pdf}{Rubin Observatory Network Pre-Verification Review Report} is the report from the panel from the June 2020 review.

The LSST Observatory Network Verification Plan defines plan for formal verification of the LSST networks.  OBSOLETE, SUPERCEDED BY LDM-732 and JIRA LVV Project

The LSST Observatory Network Verification Matrix defines requirements and methods for formal verification of the LSST networks. OBSOLETE, SUPERCEDED BY LDM-732 and JIRA LVV Project

The \href{https://confluence.lsstcorp.org/download/attachments/20284335/LSST%20Network%20O%26M%20Plan_v2.docx?version=1&modificationDate=1490879794000&api=v2}{LSST Network Operations and Management Plan}  defines the plan for operating and maintaining the LSST networks as a single integrated process.

SLAC US Data Facility Networks (\href{https://confluence.lsstcorp.org/download/attachments/20284335/Rubin-SLAC-ESNET-p1.pptx?version=1&modificationDate=1615304516000&api=v2}{ppt}, \href{https://confluence.lsstcorp.org/download/attachments/20284335/Rubin-SLAC-ESNET-p1.pdf?version=1&modificationDate=1615304502000&api=v2}{pdf})

ESnet - Europe Networks (\href{https://confluence.lsstcorp.org/download/attachments/20284335/ESnet-Europe-networks.pdf?version=1&modificationDate=1615479399000&api=v2}{pdf})

\appendix
% Include all the relevant bib files.
% https://lsst-texmf.lsst.io/lsstdoc.html#bibliographies
\section{References} \label{sec:bib}
\renewcommand{\refname}{} % Suppress default Bibliography section
\bibliography{local,lsst,lsst-dm,refs_ads,refs,books}

% Make sure lsst-texmf/bin/generateAcronyms.py is in your path
\section{Acronyms} \label{sec:acronyms}
\input{acronyms.tex}
% If you want glossary uncomment below -- comment out the two lines above
%\printglossaries





\end{document}
